\documentclass[../mn-notatki.tex]{subfiles}

\begin{document}

\section{Automatyczne różniczkowanie}

\subsection{Metoda propagacji w przód}


\subsubsection{Funkcja jednej zmiennej}
\begin{tcolorbox}
\[
\vec{x} = (x, 1, 0, \ldots) ~~~~~~~~ \vec{c} = (c, 0, 0, \ldots)
\]
\[
\vec{u} = (u, u', u'', \ldots)
\]
\[
\vec{g}(\vec{u}) = (u, u', u'', \ldots)
= (g(u), u'g'(u), u''g'(u) + (u')^2 g''(u), \ldots)
\]
\end{tcolorbox}

\begin{tcolorbox}
\textbf{Operacje na dżetach postaci $(u, u')$}
    $$(u,u') + (v,v') = (u + v, u' + v')$$
    $$(u,u') - (v,v') = (u - v, u' - v')$$
    $$(u,u') \cdot (v,v') = (u \cdot v, u \cdot v' + u' \cdot v)$$
    $$\frac{(u,u')}{(v,v')} = \left(\frac{u}{v}, \frac{u' - \frac{u}{v}\cdot v'}{v}\right)$$
    $$\sin(u,u') = (\sin u, \cos u \cdot u')$$
    $$\cos(u,u') = (\cos u, -\sin u \cdot u')$$
    $$e^{(u,u')} = (e^u, u' \cdot e^u)$$
    $$\ln(u,u') = \left(\ln u, \frac{u'}{u}\right)$$
\end{tcolorbox}

\subsubsection{Pochodne kierunkowe}
\begin{tcolorbox}
\[
f: \mathbb{R}^n \rightarrow \mathbb{R}, x, u \in \mathbb{R}^n:
g(t) = f(x + t \cdot u)
\]
\[
g'(t) = u \cdot f(t \cdot u + x) \Rightarrow g'(0) = u \cdot f(x)
\]
\[
x = ((x_1, 0), (x_2, 0), \ldots, (x_n, 0))
\]
\[
u = ((u_1, 0), (u_2, 0), \ldots, (u_n, 0))
\]
\[
t = (0,1)
\]
\end{tcolorbox}
Redukujemy problem obliczenia $\nabla_u f(x)$ do obliczenia pochodnej funkcji
jednej zmiennej.
% Poniważ chcemy policzyć pochodną w punkcie $t = 0$, zmienna $t$ zostanie
% zastąpiona parą $(0,1)$, natomiast $x = ((x_1, 0), (x_2, 0), \ldots, (x_n, 0))$,
% $u = ((u_1, 0), (u_2, 0), \ldots, (u_n, 0))$.

\subsubsection{Gradient}
Aby obliczyć pochodne cząstkowe funkcji $f$ wystarczy obliczyć
$\nabla_{e_i}f(x)$,
% ($i = 1, \ldots, n$),
gdzie $e_i$ to wektor z bazy $\mathbb{R}^n$.

\begin{tcolorbox}
\[
\vec{x} = (x, 1, 0, 0, \ldots) ~~~~~~~~ \vec{y} = (y, 0, 1, 0, \ldots) ~~~~~~~~ \vec{c} = (c, 0, 0, 0, \ldots)
\]
% \[
% \vec{u} = (u, u', u'', \ldots)
% \]
\[
\vec{f}(\vec{x}) = \left(f(x), \frac{\partial f}{\partial x}, \frac{\partial f}{\partial y}, \ldots \right)
% = (g(u), u'g'(u), u''g'(u) + (u')^2 g''(u), \ldots)
\]
\end{tcolorbox}

\begin{tcolorbox}
\textbf{Operacje na dżetach $m+1$ wymiarowych wektorów}
    $$(u,u_1,\ldots, u_m) + (v,v_1,\ldots, v_m) = (u + v, u_1 + v_1, \ldots, u_m + v_m)$$
    $$(u,u_1,\ldots, u_m) - (v,v_1,\ldots, v_m) = (u - v, u_1 - v_1, \ldots, u_m - v_m)$$
    $$(u,u_1,\ldots, u_m) \cdot (v,v_1,\ldots, v_m) = (u \cdot v, u \cdot v_1 + u_1 \cdot v, \ldots, u \cdot v_m + u_m \cdot v)$$
    $$\frac{(u,u_1,\ldots, u_m)}{(v,v_1,\ldots, v_m)} = \left(\frac{u}{v}, \frac{u_1 - \frac{u}{v}\cdot v_1}{v}, \ldots, \frac{u_m - \frac{u}{v}\cdot v_m}{v}\right)$$
    $$\sin(u,u_1,\ldots, u_m) = \text{zostawiam jako ćwiczenie dla czytelnika}$$
    $$\cos(u,u_1,\ldots, u_m) = \ldots$$
    $$e^{(u,u_1,\ldots, u_m)} = \ldots$$
\end{tcolorbox}

\end{document}
