\documentclass[../mn-notatki.tex]{subfiles}

\begin{document}

\section{Metody iteracyjne rozwiązywania równań nieliniowych}

\subsection{Rząd zbieżności}
\begin{tcolorbox}
\[
\text{liczba } p := \lim_{n \to \infty} \frac{|x_{n+1} - \alpha|}{|x_n - \alpha|^p} = c, c \neq 0
\]
\end{tcolorbox}

\subsection{Metoda bisekcji}

\begin{tcolorbox}
\textbf{Założenia}
\begin{itemize}
    \item $f$ - jest funkcją ciągłą na przedziale $[a,b]$
    \item $f(a)f(b) < 0$\\
\end{itemize}

\textbf{$k$-ty krok algorytmu}
\[
c_k = \frac{a_k + b_k}{2}
\]
\begin{gather*}
f(a_k)f(c_k) > 0 : (a_{k+1}, b_{k+1}) = (c_k, b_k)\\
f(b_k)f(c_k) > 0 : (a_{k+1}, b_{k+1}) = (a_k, c_k)
\end{gather*}
% Jeżeli $f(c_k) = 0$ to kończymy obliczenia.
\textbf{Kończymy obliczenia jeżeli:}
\begin{itemize}
    \item Osiągnięto dokładność $\delta: |b_n - a_n| < \delta$
    \item Wartość funkcji jest bliska $0$: $|f(c_n)| < \varepsilon$
    \item Wykonano $M$ iteracji \textit{(potrzebne gdy poprzednie zawiodą)}\\
\end{itemize}
\[
\Delta x = |x_k - x| \leqslant \frac{|b - a|}{2^k}
\]

Niezawodna, ale wolno zbieżna.
\end{tcolorbox}

\subsection{Regula falsi}

\begin{tcolorbox}
\textbf{Założenia}
\begin{itemize}
    \item $f$ - jest funkcją ciągłą na przedziale $[a,b]$
    \item $f(a)f(b) < 0$\\
\end{itemize}

\textbf{$k$-ty krok algorytmu}
\[
x_k = a_k - \frac{f(a_k)}{f(b_k) - f(a_k)}(b_k - a_k)
\]
\begin{gather*}
% f(x_k) = 0 \Longrightarrow \text{koniec}\\
f(x_k) < \varepsilon_{max} : \text{koniec}\\
f(x_k)f(a_k) < 0 : (a_{k+1}, b_{k+1}) = (a_k, x_k)\\
f(x_k)f(a_k) > 0 : (a_{k+1}, b_{k+1}) = (x_k, b_k)
\end{gather*}

\end{tcolorbox}

\subsection{Metoda siecznych}

\begin{tcolorbox}
\textbf{Założenia}
\begin{itemize}
    \item $f$ - jest funkcją ciągłą na przedziale $[a,b]$
    \item Rezygnujemy z założenia, że funkcja na końcach przedziału ma różne znaki\\
\end{itemize}

\textbf{$k$-ty krok algorytmu}
\[
x_{k+1} = x_k - \frac{f(x_k)(x_k - x_{k-1})}{f(x_k) - f(x_{k-1})}
\]
\end{tcolorbox}

Modyfikacja regula falsi, może być rozbieżna. Jeśli $x_n, x_{n+1}$ są już
dobrymi przybliżeniami pierwiastka to mamy zdecydowanie szybszą zbieżność ciągu
iteracji.

\subsection{Metoda Newtona \textit{(stycznych)}}

\begin{tcolorbox}
\textbf{Założenia}
\begin{itemize}
    \item $f$ - jest funkcją ciągłą na przedziale $[a,b]$
\end{itemize}

\textbf{$n$-ty krok algorytmu}
\[
x_{n+1} = x_n - \frac{f(x_n)}{f'(x_n)}
\]
\end{tcolorbox}


\subsection{Metoda Newtona dla funkcji wielu zmiennych}

\begin{tcolorbox}
\textbf{Założenia}
\[
f = (f_1, f_2, \ldots, f_n) : \mathbb{R}^n \rightarrow \mathbb{R}^n
% \text{jest różniczkowalną funkcją wielu zmiennych}
\]

\textbf{$i$-ty krok algorytmu}
\[
x_{i+1} = x_i - \left[ D f(x_i) \right]^{-1} f(x_i)
\]

\textbf{NIE NALEŻY} odwracać macierzy Jacobiego!!!
Zamiast tego obliczamy układ równań:
\[
D f(x_i) (x_{i+1} - x_i) = -f(x_i)
\]
\end{tcolorbox}

Metoda zbieżna lokalnie. Oznacza to, że
punkt początkowy powinien zostać wybrany jako już dobre przybliżenie
szukanego miejsca zerowego.

\subsection{Krokowość metody}
\begin{tcolorbox}
Jeśli metoda jest  $n$-krokowa,
oznacza to, że do wyznaczenia kolejnego punktu iteracji potrzebne jest $n$
poprzednich punktów oraz wartości badanej funkcji w tych punktach.
\end{tcolorbox}

\pagebreak
\end{document}
