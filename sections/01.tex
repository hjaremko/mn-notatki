\documentclass[../mn-notatki.tex]{subfiles}

\begin{document}

\section{Analiza błędów}

\subsection{Oznaczenia}

\begin{tcolorbox}
\textbf{Błąd bezwzględny} - wartość przybliżona minus wartość dokładana
\[
\Delta x = \bar{x} - x
\]
\end{tcolorbox}

\begin{tcolorbox}
\textbf{Błąd względny} - $\frac{\text{błąd}}{\text{wartość dokładna}}$
\[
\varepsilon_x = \frac{\Delta x}{x}
\]
\end{tcolorbox}

\begin{tcolorbox}
\textbf{Epsilon maszynowy} - dla danej arytmetyki
\[
\varepsilon_{masz} = 2^{-(D+1)}
\]
$D$ - liczba bitów używanych przez daną arytmetykę zmiennoprzecinkową do
reprezentacji mantysy (dla \texttt{double} $52$).

Epsilon maszynowy jest największą liczbą reprezentowalną taką, że
\[
\fl(1 + \varepsilon) = 1
\]
\end{tcolorbox}

\begin{tcolorbox}
\textbf{Arytmetyka zmiennoprzecinkowa}\\
Istnieje liczba $\varepsilon$ taka, że $|\varepsilon| < \varepsilon_{masz}$ oraz
\[
\fl(a \star b) = (a \star b)(1 + \varepsilon)
\]
\end{tcolorbox}


\subsection{Uwarunkowanie zadania}

\begin{tcolorbox}
\textbf{Oszacowanie na błąd wyniku}\\
\[
\Delta y = D \varphi(x) \Delta x
\]
$D \varphi(x)$ - macierz Jacobiego odwzorowania $\varphi$
\end{tcolorbox}

\begin{tcolorbox}
\textbf{Błąd względny wyniku}\\
\[
\varepsilon_{y_i} = \sum_{j=1}^{n} \frac{\partial \varphi_i(x)}{\partial x_j}
\cdot \frac{x_j}{\varphi_i(x)} \varepsilon_{x_j}
\]
\end{tcolorbox}

\begin{tcolorbox}
\textbf{Współczynniki uwarunkowania}\\
\[
k_{ij} = \frac{\partial \varphi_i(x)}{\partial x_j} \cdot \frac{x_j}{\varphi_i(x)}
\]
% \begin{itemize}
    % \item
    Określa jaki wpływ na błąd wyniku $y_i$ ma błąd $x_j$.\\
    % \item
     Jeżeli \textbf{wartość bezwzględna} któregokolwiek współczynnika
uwarunkowania jest duża to zadanie jest \textbf{źle uwarunkowane}.
% \end{itemize}

\end{tcolorbox}

\begin{tcolorbox}
\textbf{Wskaźniki uwarunkowania}\\
Liczbę $c$ nazywamy wskaźnikiem uwarunkowania zadania
\textit{(względem norm $||\cdot||_{\mathbb{R}^n}, ||\cdot||_{\mathbb{R}^m}$)}
jeżeli \textit{(dla $\varphi(x) \neq 0, x \neq 0$)}
\[
\underbrace{\frac{||\varphi(\bar{x}) - \varphi(x)||_{\mathbb{R}^m}}{||\varphi(x)||_{\mathbb{R}^m}}}_{\text{błąd względny wyniku}}
\leqslant
c  \cdot
\underbrace{\frac{||\bar{x} - x||_{\mathbb{R}^n}}{||x||_{\mathbb{R}^n}}}_{\text{błąd względny wejścia}}
\]

Dla układu równań
$Ax = b :
c = || A^{−1}|| \cdot ||A||$
\end{tcolorbox}

\subsection{Algorytm numerycznie stabilny}

\begin{tcolorbox}
\textbf{Algorytm}\\
Ciąg $A = (\varphi_0, \varphi_1, \ldots, \varphi_n)$ nazywamy algorytmem i definiujemy
\[
A(x) = \varphi_k \circ \ldots \circ \varphi_1 \circ \varphi_0(x)
\]
Algorytm $A$ jest algorytmem obliczania wartości $\varphi(x)$ jeżeli
$\varphi(x) = A(x)$.
\end{tcolorbox}

\begin{tcolorbox}
\textbf{Współczynniki wzmocnienia}\\
Jeżeli wpływ błędu zaokrągleń $\varepsilon_i$ na ostateczny błąd obliczeń
wynosi $r_i \varepsilon_i$ to liczbę $r_i$ nazywamy współczynnikiem wzmocnienia
dla $\varepsilon_i$.
\[
f(c + h) =
f(c) +
\frac{f'(c)}{1!} h +
\frac{f'(c)}{2!} h^2
\]
\[
f(\underbrace{x}_{c} + \underbrace{k_1 \varepsilon_1 + k_2 \varepsilon_2}_h) =
f(x) +
\frac{f'(x)}{1!} k_1 \varepsilon_1 +
\frac{f'(x)}{1!} k_2 \varepsilon_2 +
\underbrace{\ldots}_{\text{zaniedbujemy}} \varepsilon_1 \varepsilon_2
\]
\end{tcolorbox}


\begin{tcolorbox}
\textbf{Błąd nieunikniony}\\
\[
\Delta^0y = \left(|D\varphi|\cdot|x| + |y| \right)\varepsilon_{masz}
\]
\end{tcolorbox}

\begin{tcolorbox}
\textbf{Błąd nieszkodliwy}\\
Błąd zaokrągleń $\varepsilon_i$ nazywamy nieszkodliwym jeżeli jego wpływ na
całkowity błąd obliczeń jest co najwyżej na poziomie \textbf{błędu nieuniknionego}.\\
\[
\Delta^0y = \left(|D\varphi|\cdot|x| + |y| \right)\varepsilon_{masz}
\]
\end{tcolorbox}

\begin{tcolorbox}
\textbf{Algorytm numerycznie stabilny}\\
Gdy wszystkie jego błędy zaokrągleń są nieszkodliwe.
\[
\text{dla każdego~} |k_{ij}|\leqslant \Delta^0y
\]
\end{tcolorbox}

\begin{tcolorbox}
\textbf{Algorytm numerycznie użyteczny}\\
Algorytm $A_1$ jest numerycznie użyteczniejszy od algorytmu $A_2$, jeżeli dla
każdych danych wejściowych całkowity wpływ błędów zaokrągleń jest mniejszy
w algorytmie $A_1$.
\end{tcolorbox}

\pagebreak
\end{document}
