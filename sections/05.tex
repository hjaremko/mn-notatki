\documentclass[../mn-notatki.tex]{subfiles}

\begin{document}

\section{Metoda ortogonalizacji Gramma-Schmidta}

\begin{tcolorbox}
\textbf{Norma indukowana}
\[
||x|| = \sqrt{\langle x,x \rangle}
\]
\end{tcolorbox}

\begin{tcolorbox}
\textbf{Prostopadłość}\\
Wektory $u$ i $v$ nazywamy \textbf{ortogonalnymi} jeżeli $\langle x,y \rangle = 0$.\\
Układ wektórów $x_1, \ldots, x_k, k \geqslant 1$ nazywamy \textbf{ortonormalnymi}
jeżeli:
\begin{itemize}
    \item $||x_i|| = 1$ dla $i = 1, \ldots, k$
    \item iloczyn skalarny $\langle x_i; x_j \rangle = 0$ dla $i,j = 1, \ldots, k$ oraz $i \neq j$
\end{itemize}
\end{tcolorbox}

\begin{tcolorbox}
Macierz $Q_{n\times n}$ nazywamy \textbf{ortonormalną} \textit{(lub unitarną)}
jeżeli $Q_T \cdot Q = \mathcal{I}_n$.\\

Jeżeli macierz $Q$ jest ortonormalna, to wiersze i kolumny tej macierzy
tworzą układy wektorów ortonormalnych.\\

$Q^T$ jest macierzą odwrotną do macierzy ortonormalnej $Q$.
\end{tcolorbox}

\begin{tcolorbox}
\[
Qx = b \Rightarrow x = Q^T b
\]
\end{tcolorbox}

\subsection{Algorytm}
\begin{tcolorbox}
Zadaniem algorytmu ortonormalizacji jest przekształcenie układu
wektorów $X_1 ,X_2 ,\ldots,X_k$ w układ wektorów ortnonormalncyh $Y_1 ,Y_2 ,\ldots,Y_k$.
\end{tcolorbox}

\begin{gather*}
\hat{Y}_i = X_i - \sum_{j=1}^{i-1} \langle X_i; Y_i \rangle Y_j, \tab \text{dla~~ } i = 1, \ldots, k\\
Y_i =
\begin{cases}
\frac{\hat{Y}_i}{||\hat{Y}_i||}, \tab \text{jeżeli~~ } \hat{Y}_i \neq 0\\
0, \tab \text{~~~~~~jeżeli~~ } \hat{Y}_i = 0
\end{cases}
\end{gather*}

\pagebreak
\end{document}
