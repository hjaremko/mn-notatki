\documentclass[../mn-notatki.tex]{subfiles}

\begin{document}

\section{Metody iteracyjne rozwiązywania układów równań liniowych}


\begin{tcolorbox}
\textbf{Układ równań}
\[
Ax = b
\]
\end{tcolorbox}

Zamiast jawnego szukania rozwiązania $x$ tworzymy ciągi $x_n$ zbieżne do $x.$

\subsection{Postać macierzowa metod iteracynych}

\begin{tcolorbox}
% \textbf{Układ równań}
Niech $Q$ będzie dowloną macierzą.
\[
(Q - Q + A)x = b
\]
\[
Qx = (Q-A)x + b
\]
Zamieniamy to na metodę iteracyjną:
\[
Qx^{(n+1)} = (Q - A)x^{(n)} + b
\]
\[
x^{(n+1)} = Q^{-1}(Q - A)x^{(n)} + Q^{-1}b
\]
\end{tcolorbox}

\begin{tcolorbox}
Metoda iteracyjna jest zbieżna jeżeli ciąg $\{x^{(n)}\}$ jest zbieżny do
rozwiązania $x$ dla dowolnego punktu startowego $x_0$.
\end{tcolorbox}

Macierz $Q$ powinna spełniać:

\begin{itemize}
    \item Obliczenie $x^{(n)}$ jest szybkie \textit{(odpowiedni układ równań
    $Q y = c$ jest "łatwy" do rozwiązania)}.
    \item Ciąg  $\{x^{(n)}\}$  jest szybko zbieżny do $x$.
\end{itemize}

\pagebreak

\subsubsection{Rozkład macierzy $A$}
\begin{tcolorbox}
\[
A = L + D + U
\]
\end{tcolorbox}

\[
A
=
\begin{bmatrix}
0      & 0      & \ldots & 0\\
a_{21} & 0      & \ldots & 0\\
\vdots & \vdots & \ddots & \vdots\\
a_{n1} & a_{n2} & \ldots & 0
\end{bmatrix}
+
\begin{bmatrix}
a_{11} & 0      & \ldots & 0\\
0      & a_{22} & \ldots & 0\\
\vdots & \vdots & \ddots & \vdots\\
0      & 0      & \ldots & a_{nn}
\end{bmatrix}
+
\begin{bmatrix}
0      & a_{12} & \ldots & a_{1n}\\
0      &        & \ldots & a_{2n}\\
\vdots & \vdots & \ddots & \vdots\\
0      & 0      & \ldots & 0
\end{bmatrix}
\]


\subsection{Metoda Jacobiego}
\begin{tcolorbox}
\textbf{Macierzowo}
\[
Q = D
\]
\end{tcolorbox}

\begin{tcolorbox}
\textbf{Algorytmicznie}
\[
x_k^{n+1} = \frac{1}{a_{kk}} \left(b_k - \sum^{n}_{\substack{m=1 \\ m\neq k}}
a_{km} x_m^n \right)
\]
\end{tcolorbox}

\subsubsection{Przykład}
\begin{table}[!htb]
    \begin{minipage}{.5\linewidth}
\[
\begin{cases}
4x_1 + 2x_2 - 3x_3 = 8\\
2x_1  - 8x_2 + 5x_3 = 4\\
3x_1 - 8x_2 + 7x_3 = 5
\end{cases}
\]
    \end{minipage}%
    \begin{minipage}{.5\linewidth}
\[
\begin{cases}
x_1^{n+1} =  \frac{1}{4} ( 8 - 2x_2^n + 3x_3^n  ) \\
x_2^{n+1} = -\frac{1}{8} ( 4 - 2x_1^n - 5x_3^n  ) \\
x_3^{n+1} =  \frac{1}{7} ( 5 - 3x_1^n + 8x_2^n  )
\end{cases}
\]
    \end{minipage}%
\end{table}


\subsection{Metoda Gaussa-Seidla}
\begin{tcolorbox}
\textbf{Macierzowo}
\[
Q = L + D
\]
\[
Qx^{(n+1)} = (Q - A)x^{(n)} + b
\]
\[
(L+D)x^{(n+1)} = (-U)x^{(n)} + b
\]
\[
Dx^{(n+1)} = (-U)x^{(n)} - Lx^{(n+1)} + b
\]
\[
G = -(L+D)^{-1} U
\]
\end{tcolorbox}

\subsubsection{Przykład}
\begin{table}[!htb]
    \begin{minipage}{.5\linewidth}
\[
\begin{cases}
4x_1 + 2x_2 - 3x_3 = 8\\
2x_1  - 8x_2 + 5x_3 = 4\\
3x_1 - 8x_2 + 7x_3 = 5
\end{cases}
\]
    \end{minipage}%
    \begin{minipage}{.5\linewidth}
\[
\begin{cases}
x_1^{n+1} =  \frac{1}{4} ( 8 - 2x_2^n + 3x_3^n  ) \\
x_2^{n+1} = -\frac{1}{8} ( 4 - 2x_1^{n+1} - 5x_3^n  ) \\
x_3^{n+1} =  \frac{1}{7} ( 5 - 3x_1^{n+1} + 8x_2^{n+1}  )
\end{cases}
\]
    \end{minipage}%
\end{table}

\subsection{Metoda SOR}
\begin{tcolorbox}
\textbf{Macierzowo}
\[
Q = \frac{1}{\omega} (D + \omega L) = \frac{1}{\omega} D + L
\]
\end{tcolorbox}

\subsubsection{Przykład}
\begin{table}[!htb]
    \begin{minipage}{.5\linewidth}
\[
\begin{cases}
4x_1 + 2x_2 - 3x_3 = 8\\
2x_1  - 8x_2 + 5x_3 = 4\\
3x_1 - 8x_2 + 7x_3 = 5
\end{cases}
\]
    \end{minipage}%
    \begin{minipage}{.5\linewidth}
\[
\begin{cases}
\tilde{x}_1^{n+1} =  \frac{1}{4} ( 8 - 2x_2^n + 3x_3^n  ) \\
x_1^{n+1} =  (1 -\omega)x_1^n + \omega \tilde{x}_1^{n+1} \\
\tilde{x}_2^{n+1} = -\frac{1}{8} ( 4 - 2x_1^{n+1} - 5x_3^n  ) \\
x_2^{n+1} =  (1 -\omega)x_2^n + \omega \tilde{x}_2^{n+1} \\
\tilde{x}_3^{n+1} =  \frac{1}{7} ( 5 - 3x_1^{n+1} + 8x_2^{n+1}  ) \\
x_3^{n+1} =  (1 -\omega)x_3^n + \omega \tilde{x}_3^{n+1} \\
\end{cases}
\]
    \end{minipage}%
\end{table}

\subsection{Metoda Richardsona \textit{(poprawianie iteracyjne)}}
\begin{tcolorbox}
\textbf{Macierzowo}
\[
Q = I
\]
\end{tcolorbox}

\subsection{Twierdzenia o zbieżności metod iteracyjnych}

\begin{tcolorbox}
\[
x^{k+1} = G x^k + c   ~~~~~~~(\star)
\]
\end{tcolorbox}

\begin{tcolorbox}
Jeżeli $||G|| < 1$ dla pewnej normy macierzowej to metoda iteracyjna $(\star)$
jest zbieżna.
\end{tcolorbox}

\begin{tcolorbox}
Metoda iteracyjna $(\star)$ jest zbieżna wtedy i tylko wtedy gdy promień
spektralny macierzy $G$ jest mniejszy od $1$.
\[
\rho(G) < 1
\]
\end{tcolorbox}

\begin{tcolorbox}
Dla układu równań $Ax = b$. Jeżeli macierz $A$ jest
\textbf{dominująca przekątniowo}
to metody Jacobiego i Gaussa-Seidla są zbieżne.
\end{tcolorbox}

\begin{tcolorbox}
Jeżeli
\begin{itemize}
    \item $A$ jest hermitowska, dodatnio określona,
    \item macierz $Q = \frac{1}{\omega} D + L$ jest nieosobliwa,
    \item $\omega$ należy do przedziału $(0,2)$
\end{itemize}
to metoda $SOR$ jest zbieżna.
\end{tcolorbox}

\pagebreak
\end{document}
