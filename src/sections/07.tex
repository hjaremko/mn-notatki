\documentclass[../mn-notatki.tex]{subfiles}

\begin{document}

\section{Zagadnienia własne}

\subsection{Definicje}
\begin{tcolorbox}
Liczbę $\lambda \in \mathbb{C}$ nazywamy \textbf{wartością własną}
\underline{macierzy kwadratowej} $A \in \mathbb{C}^{n\times n}$, jeżeli
istnieje wektor $v \in \mathbb{C}^n$ taki, że $Av = \lambda v$.\\

Każdy wektor $v \neq 0$ spełniający równość $Av = \lambda v$ nazywamy
\textbf{wektorem własnym} macierzy $A$ odpowiadającym wartości własnej
$\lambda$.\\

Zbiór wszystkich wartości własnych macierzy kwadratowej $A$ nazywamy
\textbf{widmem macierzy $A$} \textit{(lub spektrum macierzy $A$)}
i oznaczamy $\sigma(A)$.\\

\textbf{Promień spektralny} macierzy $A$ \textit{(oznaczamy $\rho(A)$)}
definiujemy jako
\[
\rho(A) := \max\{ |\lambda| : \lambda \in \sigma(A) \}
\]
\end{tcolorbox}
\begin{tcolorbox}
Jeżeli $v$ jest wektorem własnym dla $\lambda$, to każdy wektor postaci
$\alpha v$, gdzie $\alpha \in \mathbb{R} \backslash \{0\}$ jest wektorem
własnym macierzy $A$.
\end{tcolorbox}

\subsection{Metoda dokładna}
\begin{tcolorbox}
\[
W_A(\lambda) = \det(A - \lambda \mathcal{I})
\]
nazywamy \textbf{wielomianem charakterystycznym} macierzy $A$.\\
Stopień tego wielomianu jest równy rozmiarowi tej macierzy.
\end{tcolorbox}
\begin{tcolorbox}
Wszystkie wartości własne macierzy $A$ są pierwiastkami jej wielomianu
charakterystycznego, czyli $\sigma(A) = W_A^{-1}\left(\{0\}\right)$.
\end{tcolorbox}
\begin{tcolorbox}
Wyznaczanie wartości własnych jest tak samo trudne jak wyznaczanie pierwiastków
wielomianu.
\end{tcolorbox}

\subsubsection{Przykład}

Wyznacz wartości własne macierzy $A = \left[ \begin{smallmatrix}
1 & 2 & 1\\
2 & 1 & 1\\
2 & -1 & 0
\end{smallmatrix} \right]$.

Zgodnie z definicją wielomian charakterystyczny $A$ to
\[
W_A(\lambda) = \det(A - \lambda \mathcal{I}) =
\left[ \begin{smallmatrix}
1 & 2 & 1\\
2 & 1 & 1\\
2 & -1 & 0
\end{smallmatrix} \right]
=
-\lambda^3 + 2\lambda^2 + 6\lambda
\]
Pierwiastki tego wielomianu to $\lambda_1 = 0, \lambda_2 = 1 - \sqrt{7},
\lambda_3 = 1 + \sqrt{7}$. \\
Na mocy twierdzenia są to wartości własne macierzy $A$.

\subsection{Metoda potęgowa}

\begin{tcolorbox}
Metoda potęgowa pozwala wyznaczyć \textbf{dominującą wartość własną}, czyli
o największym module i odpowiadający jej wektor własny.\\

Dla macierzy $A$ i wektor początkowego $x_0$ wyznaczamy
\[
x_k = A^k x_0
\]
który przy pewnych założeniach jest zbieżny do wektora własnego odpowiadającego
dominującej wartości własnej.
\end{tcolorbox}
\begin{tcolorbox}
Działa poprawnie jeżeli:
\begin{itemize}
    \item Wartość o największym module jest odseparowana od pozostałych
    \textit{(ich moduły są wyraźnie mniejsze)}.
    \item Macierz ma prostą strukturę tzn. wektory własne istnieją i tworzą
    bazę przestrzeni.
\end{itemize}
\end{tcolorbox}

\subsubsection{Algorytm}
\begin{tcolorbox}
\textbf{Wejście:} $A, x_0, n$\\
Dla $k$ od $1$ do $n$ powtarzaj
\begin{itemize}
    \item $y_{k-1} = \frac{x_{k-1}}{||x_{k-1}||}$
    \item $x_k = A y_{k-1}$
    \item $r_k = \frac{\varphi(x_k)}{\varphi(x_{k-1})}$
\end{itemize}
\textbf{Wyjście:}
\begin{itemize}
    \item $r_n$ - przybliżenie największej wartości własnej $A$
    \item $x_k$ - przybliżenie wektora własnego
\end{itemize}
\vskip 0.3cm
Wybór funkcjonału ($x = (x_1, x_2, \ldots, x_n)$)
\begin{itemize}
    \item $\varphi(x) = x_i$
    \item $\varphi_k(x) = \langle x, x_{k-1} \rangle$
\end{itemize}

\end{tcolorbox}


\subsection{Odwrotna metoda potęgowa}
\begin{tcolorbox}
Jeżeli $\lambda$ jest wartością własną nieosobliwej macierzy $A$, to
$\lambda^{-1}$ jest wartością własną macierzy $A^{-1}$.
\end{tcolorbox}

Jeżeli dla macierzy $A$ mamy
\[
|\lambda_1| \geqslant |\lambda_2| \geqslant \ldots \geqslant
|\lambda_{n-1}|  > |\lambda_n| > 0
\]
to
\[
|\lambda_1^{-1}| \leqslant |\lambda_2^{-1}| \leqslant \ldots \leqslant
|\lambda_{n-1}^{-1}| < |\lambda_n^{-1}|
\]

Zatem aby wyznaczyć $\lambda_n^{-1}$ można zastosować metodę potęgową dla
macierzy $A^{-1}$.

\begin{tcolorbox}
Nie należy obliczać $A^{-1}$.\\

Zamiast obliczać $x_{k+1} = A^{-1} x_k$ rozwiązujemy układ równań
\[
Ax_{k+1} = x_k
\]

Dobrze jest najpierw rozłożyć macierz $A$ np. na $LU$ i wykorzystać to do
rozwiązywania układu równań.
\end{tcolorbox}


\subsection{Wartość własna najbliższa $\mu$}
\begin{tcolorbox}
Jeżeli $\lambda$ jest wartością własną nieosobliwej macierzy $A$, to
$\lambda - \mu$ jest wartością własną macierzy $A - \mu\mathcal{I}$.
\end{tcolorbox}

% \begin{tcolorbox}
Stosując odwrotną metodę potęgową dla macierzy $A - \mu\mathcal{I}$ otrzymujemy
$\beta$ \textit{(najmniejszą wartość własną tej macierzy)}.\\

Wtedy $\lambda = \mu + \beta$ jest wartością własną $A$ najbliższą $\mu$.
% \end{tcolorbox}

\begin{tcolorbox}
Zamiast obliczać $x_{k+1} = (A - \mu\mathcal{I})^{-1} x_k$ rozwiązujemy układ równań
\[
(A - \mu\mathcal{I}) x_{k+1} = x_k
\]
\end{tcolorbox}


\subsection{Metoda QR}

\begin{tcolorbox}
\textbf{Wejście:} macierz $A$, ilość iteracji $M$ \\
Dla $k$ od $1$ do $M$ wykonujemy:
\begin{itemize}
    \item $(Q_k, R_k) := $ rozkład $QR$ macierzy $A_k$
    \textit{(gdzie $Q$ - unitarna)}
    \item $A_{k + 1} := R_kQ_k$
\end{itemize}
\end{tcolorbox}

\begin{itemize}
    \item Przy pewnych założeniach elementy poniżej przekątnej w macierzach
    $A_k$ zmierzają do zera.
    \item Elementy przekątniowe dążą do wartości własnych.
    \item W przypadku zespolonych wartości własnych $\alpha \pm i \beta$
    na przekątnej pojawią się klatki $2 \times 2$ postaci $
    \left[ \begin{smallmatrix}
    \alpha & \beta\\
    -\beta & \alpha
    \end{smallmatrix} \right]
    $.
\end{itemize}


\subsection{Macierze Hessenberga}

Aby przyśpieszyć zbieżność algorytmu $QR$ najpierw sprowadza się macierz
do postaci Hessenberga.

\begin{tcolorbox}
Macierz $H = [h_{ij}]$ nazywamy \textbf{macierzą Hessenberga}, jeżeli
$h_{ij} = 0$ dla $j > i + 1$, czyli jest ona postaci
\[
    H = \left[ \begin{smallmatrix}
    \star & \star & \star & \star & \star & \star\\
    \star & \star & \star & \star & \star & \star\\
    0 & \star & \star & \star & \star & \star\\
    0 & 0 & \star & \star & \star & \star\\
    0 & 0 & 0 & \star & \star & \star\\
    0 & 0 & 0 & 0 & \star & \star
    \end{smallmatrix} \right]
\]
\end{tcolorbox}

\begin{tcolorbox}
Jeśli $A$ jest macierzą kwadratową, to istnieje macierz ortogonalna $Q$ oraz
macierz Hessenberga $H$ takie, że
\[
A = Q^T H Q
\]
W szczególności macierze $A$ oraz $H$ mają takie same wartości własne.
\end{tcolorbox}

\subsubsection{Algorytm redukcji pierwszej kolumny macierzy do postaci
Hassenberga}

\begin{tcolorbox}
$a_{k1}$ = element głowny spośród $a_{i1}$ dla $i > 2$\\
Zamiana wierszy $w_2$ i $w_k$ oraz kolumnd $k_2$ i $k_k$.
Dla $i$ od $3$ do $n$ wykonaj:
\begin{itemize}
    \item $s := \frac{a_{i1}}{a_{21}} $
    \item $w_i := w_i - s \cdot w_2$
    \item $k_2 := k_2 + s\cdot k_i$
\end{itemize}
\end{tcolorbox}

Po doprowadzeniu do postaci Hessenberga pierwszej kolumny, wywołujemy ten
algorytm dla podmacierzy bez pierwszego wiersza i pierwszej kolumny itd.

\subsection{Metoda QR z przesunięciami}

\begin{tcolorbox}
\textbf{Wejście:} $A$, $M$ \\
$A_1 := $ przekształcenie macierzy $A$ do postaci Hessenberga
Dla $k$ od $1$ do $M$ wykonujemy:
\begin{itemize}
    \item $(Q_k, R_k) := $ rozkład $QR$ macierzy $(A_k - z_k\mathcal{I})$
    \item $A_{k + 1} := R_kQ_k + z_k\mathcal{I}$
\end{itemize}
\end{tcolorbox}

Dobrze wybrane przesunięcie znacznie przyspiesza zbieżność metody.
$z_k$ powinno być dobrym przybliżeniem jednej z wartości własnych.
Jeżeli w ostatnim wierszu poza elementem diagonalnym mamy same 'prawie' zera
to dokonujemy deflacji macierzy i dalsze obliczenia wykonujemy dla macierzy
bez ostatniego wiersza i ostatniej kolumny.

\subsection{Macierze podobne}

\begin{tcolorbox}
Macierze $A, B \in \mathbb{C}^{n \times n}$ są \textbf{podobne} jeżeli
istnieje nieosobliwa macierz $P \in \mathbb{C}^{n \times n}$ taka, że
\[
B = P A P^{-1}
\]
\end{tcolorbox}

\begin{tcolorbox}
Macierze podobne mają te same zbiory wartości własnych.
\end{tcolorbox}


\subsection{Twierdzenie Gerszgorina}

\begin{tcolorbox}
Niech $A = [a_{ij}]_{i,j = 1,\ldots, n}$ będzie macierzą kwadratową.
\begin{enumerate}
    \item Jeżeli $C_i$ oznaczają koła domknięte na płaszczyźnie zespolonej
    o środkach w punktach $a_{ii}$ i promieniach równych sumie modułów
    elementów z danego wiersza spoza diagonali, tzn.
    \[
    C_i = \{ z \in \mathbb{C} : |z - a_{ii}| \leqslant
    \sum_{\substack{j = 1\\ j \neq i}}^{n} |a_{ij}| \}
    \Longrightarrow
    \sigma(A) \subset \bigcup_{i=1}^n C_i
    \]
    Dla $i = 1, \ldots, n$.

    \item Jeżeli $k$ kół $C_i$ tworzy zbiór rozłączny z pozostałymi kołami,
    to w zbiorze tym leży dokładnie $k$ wartości własnych macierzy $A$.

    \item Jeżeli do wyznaczenia kół zamiast wierszy użyjemy kolumn to
    analogiczne twierdzenie jest prawdziwe.

    \[
    C'_i = \{ z \in \mathbb{C} : |z - a_{ii}| \leqslant
    \sum_{\substack{j = 1\\ j \neq i}}^{n} |a_{ij}| \}
    \Longrightarrow
    \sigma(A) \subset \bigcup_{i=1}^n C'_i
    \]
\end{enumerate}
\end{tcolorbox}

\subsubsection{Przykład}

Wyznacz koła Gerszgorina dla macierzy $A = \left[ \begin{smallmatrix}
5 & 1 & 0\\
1 & 7 & 2\\
-1 & -1 & 3
\end{smallmatrix} \right]$.
\begin{gather*}
C_1 = \{ z \in \mathbb{C} : |z - 5| \leqslant |a_{12}| + |a_{13}| = 1 + 0 = 1 \} \\
C_2 = \{ z \in \mathbb{C} : |z - 7| \leqslant |a_{21}| + |a_{23}| = 1 + 2 = 3 \} \\
C_3 = \{ z \in \mathbb{C} : |z - 3| \leqslant |a_{31}| + |a_{33}| = 1 + 1 = 2 \} \\
\end{gather*}
Wartości własne macierzy $A$ to
\begin{gather*}
\lambda_1 = 5 \in C_1, C_2, C_3\\
\lambda_2 = 5 + \sqrt{3} \in C_2\\
\lambda_3 = 5 - \sqrt{3} \in C_2\\
\end{gather*}


\pagebreak
\end{document}
