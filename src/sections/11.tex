\documentclass[../mn-notatki.tex]{subfiles}

\begin{document}

\section{Różniczkowanie i całkowanie numeryczne}

Aby obliczyć $\int_a^b f(x) \dif x$
\begin{enumerate}
    \item Aproksymujemy funkcję $f$ np. wielomianem, funkcją sklejaną.
    \item Całkujemy aproksymację.
\end{enumerate}

\subsection{Kwadratura}

\begin{tcolorbox}
Załóżmy, że ustalony jest podział przedziału $[a,b]$
\[
a \leqslant x_0 < x_1 < x_2 < \ldots < x_N \leqslant b
\]
Wyrażenie postaci
\[
S(f) = \sum_{k=0}^{N} A_k f(x_k)
\]
nazywamy \textbf{kwadraturą} jeżeli współczynniki $A_k$ nie zależą od funkcji
podcałkowej $f: [a,b] \rightarrow \mathbb{R}$, a jedynie od punktów podziału
$x_0, \ldots, x_N$.\\
Argumenty $x_k$ nazywamy \textbf{węzłami kwadratury}.
\end{tcolorbox}

\subsection{Kwadratury interpolacyjne}

\begin{tcolorbox}
\[
p(x) = \sum_{k=0}^n f(x_k)\Phi_k(x) \tab \Phi_k(x) = \prod_{j \neq k}
\frac{x - x_j}{x_k - x_j}
\]
\[
A_k = \int_a^b \Phi_k(x) \dif x
\]
\end{tcolorbox}


% TODO
% \subsubsection{Błąd kwadratury interpolacyjnej}

\pagebreak
\subsection{Kwadratury Newtona-Cotesa}
Gdy węzły są równo odległe, to znaczy
\[
x_k = a + kh \tab h = \frac{b-a}{n}
\]
kwadraturę na nich opartną nazywamy \textbf{kwadraturą Newtona-Cotesa}.


\subsubsection{Metoda trapezów}
Funkcję interpolujemy funkcją liniową na przedziale $[a,b]$.
\[
x_0 = a \tab x_1 = b
\]
\[
\Phi_0(x) = \frac{x-b}{a-b}
\tab
\Phi_1(x) = \frac{x-a}{a-a}
\]
\[
A_0 = \int_a^b \Phi_0(x) \dif x = \frac{1}{2}(b - a)
\tab
A_1 = \int_a^b \Phi_1(x) \dif x = \frac{1}{2}(b - a)
\]

\begin{tcolorbox}
\[
\int_a^b f(x) \dif x \approx \frac{1}{2}\left(
b - a
\right) \left(
f(a) + f(b)
\right)
\]
\end{tcolorbox}

\subsubsection{Złożona metoda trapezów}

Dzielimy przedział $[a,b]$ na $N$ podprzedziałów
$a = x_0 < x_1 < x_2 < \ldots < x_N = b$ o \textbf{tej samej długości}
$h = \frac{b-a}{N}$ i w każdym z nich stosujemy metodę trapezów.

\begin{tcolorbox}
\[
\int_a^b f(x) \dif x
\approx S(f)
= h \left(
\frac{1}{2}f(x_0)
+ \sum_{i=1}^{N-1} f(x_i)
+ \frac{1}{2} f(x_N)
\right)
\]
\end{tcolorbox}

\subsubsection{Wzór Simpsona \textit{(parabol)}}

\begin{tcolorbox}
\[
S(f) = \frac{b-a}{6} \left(
f(a) + 4f \left(
\frac{a+b}{2}
\right)
+ f(b)
\right)
\]
\end{tcolorbox}

\subsubsection{Złożony wzór Simpsona}
Powstaje przez zastosowanie wzoru Simpsona  do $n$ podprzedziałów jednakowej
długości.
\begin{tcolorbox}
\[
x_i = a + ih ~~~~(0 \leqslant i \leqslant 2n)
, \tab
h := \frac{b-a}{2n}
\]
\[
S(f) =
\frac{1}{3} h \left(
f(x_0)
+ 4 \sum_{i=1}^n f(x_{2i-1})
+ 2 \sum_{i=1}^{n-1} f(x_{2i})
+ f(x_{2n})
\right)
\]
\end{tcolorbox}

\subsubsection{Wzór \textit{trzech ósmych}}

Punkty podziału przedziału całkowania
$[a,b]$,
na $n$ równych części w tym przypadku
oznaczamy przez $x_0, x_3, x_6, \ldots, x_{3n-3}, x_{3n}$.
Wobec tego podprzedziały będą postaci
$[x_0, x_3], [x_3, x_6], \ldots, [x_{3n-3}, x_{3n}]$.
Wówczas potrzebne w metodzie $\frac{3}{8}$ Newtona węzły obliczymy jako
$x_0 = a$ oraz $x_i = a + ih$, dla $1, 2, \ldots, 3n$, gdzie
$h = \frac{b-a}{3n}$.

\begin{tcolorbox}
\[
S(f) = \frac{3h}{8}
\left(
y_0
+ 3 \sum_{k=1}^n (y_{3k-2} + y_{3k-1})
+ 2 \sum_{k=1}^{n-1} (y_{3k} + y_{3n})
\right)
\]
\end{tcolorbox}


\subsection{Rząd kwadratury}

\begin{tcolorbox}
Mówimy, że kwadratura jest \textbf{rzędu $r$}, jeżeli
\begin{itemize}
    \item kwadratura jest zgodna dla wszystkich wielomianów o stopniu mniejszym
    niż $r$, czyli wartość policzonej całki za pomocą kwadratury zgadza się
    z dokładną całką takiego wielomianu,
    \item istnieje wielomian stopnia dokładnie $r$, dla którego kwadratura nie
    jest zgodna.
\end{itemize}
\end{tcolorbox}

Kwadratury Newtona-Cotesa mają rząd
\begin{itemize}
    \item $N+1$ dla $N$ niepatrzystych
    \item $N+2$ dla $N$ parzystych
\end{itemize}
lub ogólnie $2 \floor{\frac{N}{2}} + 2$, gdzie $\floor{a}$ oznacza cechę liczby
$a$.

\subsection{Kwadratury Gaussa-Legendre'a}

\begin{tcolorbox}
Dla $N+1$ węzłów istnieje dokładnie jedna kwadratura rzędu $2(N+1)$.
Nazywana jest \textbf{kwadraturą Gaussa-Legendre'a}.
\end{tcolorbox}

Dla przedziału $[-1,1]$ węzłami tej kwadratury są miejsca zerowe
\textbf{ortogonalych wielomianów Legendre'a} zdefiniowanych następująco
\[
P_n(x) = \frac{1}{2^n n!} \left((x^2-1)^n\right)^n
\]
Współczynniki tej kwadratury wyrażają się następująco
\[
A_k = \frac{-2}{(N+2)P_{N+2}(x_k)P'_{N+1}(x_k)}
\]
gdzie $x_0, \ldots, x_n$ są miejscami zerowymi wielomianu $P_{N+1}$.

Wyznaczymy wielomiany Legendre'a i węzły dla małych wartości $N$. Mamy
\begin{align*}
&P_1(x) = \frac{1}{2^1 1!} \left( x^2 - 1 \right)' = x \\
&P_2(x) = \frac{1}{2^2 2!} \left( (x^2 - 1)^2 \right)'' = \frac{1}{2}\left(3x^2 -1 \right)\\
&P_3(x) = \frac{1}{2^3 3!} \left( (x^2 - 1)^3 \right)''' = \frac{1}{2}\left(5x^2 -3 \right)\\
&P_4(x) = \frac{1}{2^4 4!} \left( (x^2 - 1)^4 \right)'''' = \frac{1}{8}\left(35x^4 -30x^2 + 3 \right)
\end{align*}
Zgodnie z podaną procedurą, węzły dla ustalonego $N$ to pierwiaski wielomianu
$P_{N+1}$. Stąd np. dla $N=2$ węzłami będą miejsca zerowe $P_3$, czyli
$\left\{ 0, \pm \sqrt{\frac{3}{5}} \right\}$. Wartości współczyników to
\[
A_k = \frac{-2}{(2+2)P_{2+2}(x_k)P'_{2+1}(x_k)}
\]
czyli
\[
A_0 = A_2 = \frac{5}{9}, \tab A_1 = \frac{8}{9}
\]

Węzły i współczynniki dotyczą całek na przedziale $[-1,1]$. Można je wykorzystać
do obliczania całem na dowolnym przedziale stosując proste podstawienie
liniowe
\[
\int_a^b f(t) \dif t = \frac{b-a}{2} \int_{-1}^1 g(x) \dif x
\]
\[
t = \frac{a+b}{2} + \frac{b-a}{2}x
\]
\[
g(x) = f(t) = f \left(\frac{a+b}{2} + \frac{b-a}{2}x \right)
\]

% TODO
% \subsection{Metody Monte Carlo}

\pagebreak
\end{document}
