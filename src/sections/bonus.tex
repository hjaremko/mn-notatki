\documentclass[../mn-notatki.tex]{subfiles}

\begin{document}

\section{Dodatek}

\begin{tcolorbox}
\textbf{Iloczyn skalarny}
\[
\langle a, b \rangle = a_1 b_1 + \ldots + a_n b_n
\]
\end{tcolorbox}


\begin{tcolorbox}
\textbf{Mnożenie macierzy}\\
Mnożymy wiersze pierwszej macierzy przez kolumny drugiej macierzy.

\[
{\begin{bmatrix}1&0&2\\-1&3&1\end{bmatrix}} {\begin{bmatrix}3&1\\2&1\\1&0\end{bmatrix}}=
{\begin{bmatrix}1\cdot 3+0\cdot 2+2\cdot 1&1\cdot 1+0\cdot 1+2\cdot 0\\
-1\cdot 3+3\cdot 2+1\cdot 1&-1\cdot 1+3\cdot 1+1\cdot 0\end{bmatrix}}=
{\begin{bmatrix}5&1\\4&2\end{bmatrix}}
\]
\end{tcolorbox}

\begin{tcolorbox}
\textbf{Odwracanie macierzy diagonalnej}
\[
\begin{bmatrix}
a_{11} & 0 & \ldots & 0\\
0 & a_{22} & \ldots & 0\\
\vdots & \vdots & \ddots & \vdots\\
0 & 0 & \ldots & a_{nn}
\end{bmatrix}^{-1}
=
\begin{bmatrix}
\frac{1}{a_{11}} & 0 & \ldots & 0\\
0 & \frac{1}{a_{22}} & \ldots & 0\\
\vdots & \vdots & \ddots & \vdots\\
0 & 0 & \ldots & \frac{1}{a_{nn}}
\end{bmatrix}
\]
\end{tcolorbox}

\begin{tcolorbox}
\textbf{Wyznacznik macierzy $3 \times 3$}
\[
\begin{bmatrix}
a_{11} & a_{12} & a_{13}\\
a_{21} & a_{22} & a_{23}\\
a_{31} & a_{32} & a_{33}
\end{bmatrix}
=
\substack{a_{11}a_{22}a_{33}+a_{12}a_{23}a_{31}+a_{13}a_{21}a_{32}-\\
(a_{13}a_{22}a_{31}+a_{11}a_{23}a_{32}+a_{12}a_{21}a_{33})
}
\]
\end{tcolorbox}

\begin{tcolorbox}
\textbf{Metoda Laplace'a}
\[
\det A =
\begin{bmatrix}
a_{11} & a_{12} & \ldots &a_{13}\\
a_{21} & a_{22} & \ldots &a_{23}\\
\ldots & \ldots & \ldots &\ldots\\
a_{a1} & a_{a2} & \ldots &a_{nn}
\end{bmatrix}
=
\sum_{j=1}^{n} a_{ij} \cdot (-1)^{i+j} M_{ij}
\]
\end{tcolorbox}

\begin{tcolorbox}
\textbf{Norma Euklidesowa}
\[
||x||_2 = \sqrt{\sum x_i^2}
\]
\end{tcolorbox}

\begin{tcolorbox}
\textbf{Norma jedynkowa}
\[
||x||_1 = \sum |x_i|
\]
\[
||A||_1 = \max_{j = 1, \ldots, n} \sum_{i = 1}^{n} |a_{ij}|
\]
\end{tcolorbox}

\begin{tcolorbox}
\textbf{Norma maksimum}
\[
||x||_\infty = \max_{i = 1, \ldots, n} |x_i|
\]
\[
||A||_\infty = \max_{i = 1, \ldots, n} \sum_{j = 1}^{n} |a_{ij}|
\]
\end{tcolorbox}

\begin{tcolorbox}
\textbf{Macierz dominująca przekątniowo}\\
Macierz dominująca to macierz, której wartości bezwzględne elementów na głównej
przekątnej są większe od sumy wartości bezwzględnych pozostałych elementów w
wierszach.
\[
{\displaystyle {\begin{bmatrix}\mathbf {10} &2&0&-1\\1&\mathbf {8} &2&2\\3&-2&\mathbf {-14} &1\\0&-1&4&\mathbf {6} \end{bmatrix}}}
\]
\end{tcolorbox}

\begin{tcolorbox}
\textbf{Wzór Hornera}\\
\[
H
\]
\end{tcolorbox}

\pagebreak
\end{document}
