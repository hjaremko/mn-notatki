\documentclass[../mn-notatki.tex]{subfiles}

\begin{document}

\section{Interpolacja wielomianowa funkcji}

\subsection{Definicje}
\begin{tcolorbox}
\end{tcolorbox}

\subsection{Wzór interpolacyjny Newtona}
\begin{tcolorbox}
\[
p_n(x) = \sum_{k=0}^{n} a_k \prod_{j=0}^{k-1} (x-x_j)
\]
\[
a_k = \frac{y_k = p_{k-1}(x_k)}{\prod_{j=0}^{k-1}(x_k - x_j)}
\]
\end{tcolorbox}

\begin{itemize}
    \item \textbf{Dodanie kolejnych punktów nie narusza współczynników już
    policzonych.}
    \item Zmiena kolejności węzłów \textbf{zmienia} postać wzoru Newtona.
    \item Zmiana wartości $y_0$ wymaga przeliczenia wszystkiego od początku.
    \item \textbf{Nie należy przekształcać wielomianu}, lecz wyznaczyć wartość
    $p_n(x)$ stosując \textbf{wzór Hornera}.
    \item W praktyce współczynniki wyznacza się stosując \textbf{metodę ilorazów
    różnicowych}.
\end{itemize}

\subsection{Wzór interpolacyjny Lagrange'a}
\begin{tcolorbox}
\[
p(x) = \sum_{k=0}^{n} y_k L_k(x)
\]
\[
L_k(x) = \prod_{\substack{j=0\\j\neq k}}^{n} \frac{x-x_i}{x_k-x_j}
\]
\end{tcolorbox}

\begin{itemize}
    \item Nie należy przekształcać wielomianu Lagrange'a do postaci kanonicznej.
    Obliczenia wartości pośrednich dokonujemy ewaluując odpowiednie wielomiany
    $L_k$.
    \item Dla ustalonyh węzłów $x_i$ wielomiany $L_j$ są stałe, zmiana
    kolejności węzłów też ich nie zmienia.
    \item Iterpolacja Lagrange'a jest zwłaszcza przydatna gdy \textbf{węzły
    się nie zmieniają}, a zmieniają się tylko wartości.
\end{itemize}

\subsection{Ilorazy różnicowe}
\subsection{Metoda Newtona z ilorazami różnicowymi}
\subsection{Zjawisko Runge'go}
\subsection{Zbieżność wielomianów iterpolacyjnych}
\subsection{Błąd inerpolacji wielomianowej}
\subsection{Interpolacja Hermite'a}
\subsection{Uogólnienie ilorazów różnicowych}


\pagebreak
\end{document}
