\documentclass[../mn-notatki.tex]{subfiles}

\begin{document}

\section{Aproksymacja}

% Daną funckję $f$ \textit{(znaną lub określoną tablicą wartości)}
% aproksymujemy funkcją $g$ z pewnej ustalonej podprzestrzeni $G$ tak, aby
% zminimalizować $||f-g||$.

\subsection{Aproksymacja jednostajna}

\begin{tcolorbox}
\[
||f-g|| = \sup_{x \in [a,b]} |f(x) - g(x)|
\]
\end{tcolorbox}

\subsection{Metoda najmniejszych kwadratów}

\begin{tcolorbox}
Jest używana, gdy wartości znane są tylko w wybranych punktach $x_i$.
Wtedy minimalizujemy względem pseudonormy
dla $i = 1, \ldots, N$.
\[
||f-g|| = \sum_{i=0}^{N} w(x_i)\left(f(x_i) - g(x_i)\right)^2, \tab w(x_i) \geqslant 0
\]
\end{tcolorbox}

\subsection{Aproksymacja średniokwadratowa}

\begin{tcolorbox}
\[
||f-g|| = \int_a^b w(x)\left(f(x) - g(x)\right)^2 \dif x
\]
Gdzie $w$ jest ciągłą, nieujemną funkcją wagową, dodatnią poza zbiorem miary zero.
\end{tcolorbox}

\begin{tcolorbox}
Ciąg $\{ p_n \}$ wielomianów ortogonalnych standardowych w przedziale
$[a, b]$ z wagą $w$ można wyznaczyć ze wzorów
\[
p_0(x) = 1
, \tab
p_1(x) = x - a_1
\]
\[
p_n(x) = (x-a_n)p_{n-1}(x) - b_n p_{n-2}(x) \tab (n \geqslant 2)
\]
gdzie dla iloczynu sklarnego $\langle f, g \rangle := \int_a^b f(x)g(x)w(x) \dif x$
jest
\[
a_n := \frac{\langle xp_{n-1}, p_{n-1} \rangle}{\langle p_{n-1}, p_{n-1} \rangle},
\tab b_n := \frac{\langle xp_{n-1}, p_{n-2} \rangle}{\langle xp_{n-2}, p_{n-2} \rangle}
\]
\end{tcolorbox}

\pagebreak

Dane funkcje bazowe \textit{(liniowo niezależne)}
$\phi_0, \phi_1, \ldots, \phi_m$. W przypadku bazy wielomianowej mogą to być
$1, x, x^2, \ldots, x^m$. Załóżmy, że mamy dane pewnie punkty
\[
\left\{ (x_0, y_0), (x_1, y_1), \ldots, (x_n, y_n) \right\}
\]

Ogólne zadanie aproksymacji polega na znalezieniu funkcji
\[
S(x) = a_0 \phi_0 + a_1 \phi_1 + \ldots + a_m \phi_m
\]
minimalizującej pewnie wyrażenie.

\begin{tcolorbox}
\[
\begin{bmatrix}
g_{00} & \ldots & g_{0m}\\
\vdots & \ddots & \vdots\\
g_{m0} & \ldots & g_{mm}
\end{bmatrix}
\cdot
\begin{bmatrix}
a_0\\
\vdots \\
a_m
\end{bmatrix}
=
\begin{bmatrix}
\rho_0\\
\vdots \\
\rho_m
\end{bmatrix}
\]

\[
g_{ik} = \sum_{j=0}^n \phi_i(x_j)\phi_k(x_j)
\]
\[
\rho_i = \sum_{j=0}^{n} \phi_i (x_j) y_j
\]
\end{tcolorbox}

\subsubsection{Przykład}
Wyznacz funkcję liniową aproksymującą punkty $(1,1), (2, 2.5), (3, 3.5), (4,4)$.\\
Używamy bazy $\phi_0(x) = 1, \phi_1(x) = x$.
\begin{align*}
g_{00} &=  \left(  \phi_0(1) \right)^2 + \left(  \phi_0(2) \right)^2
+ \left(  \phi_0(3) \right)^2 + \left(  \phi_0(4) \right)^2\\
&= 1^2 + 1^2 + 1^2 + 1^2 = 4 \\
g_{01} &= \phi_0(1)\phi_1(1) + \phi_0(2)\phi_1(2)
+ \phi_0(3)\phi_1(3) + \phi_0(4)\phi_1(4)\\ &= 1 + 2 +3 +4 = 10%\\
% \ldots
\end{align*}
\begin{align*}
\rho_{0} &=  \phi_0(1)y_0 + \phi_0(2)y_1 + \phi_0(3)y_2 + \phi_0(4)y_3\\
&= 1 \cdot 1 + 1 \cdot 2.5 + 1 \cdot 3.5 + 1 \cdot 4 = 11 \\
\rho_{1} &=  \phi_1(1)y_0 + \phi_1(2)y_1 + \phi_1(3)y_2 + \phi_1(4)y_3\\
&= 1 \cdot 1 + 2 \cdot 2.5 + 3 \cdot 3.5 + 4 \cdot 4 = 32.5
\end{align*}
\[
S(x) = a_0\phi_0(x) + a_1\phi_1(x) = 0.25 + x
\]


\subsection{Aproksymacja trygonometryczna}

Jeżeli odległość pomiędzy węzłami aproksymacji jest
zawsze taka sama oraz aproksymowana funkcja jest okresowa, to można
użyć ortogonalnego układu funkcji trygonometrycznych. Wtedy macierz
układu $G \cdot A = \mathrm{P}$ jest diagonalna i układ ten można bardzo szybko
rozwiązać.\\

\begin{tcolorbox}
Ustalmy pewną liczbę naturalną $L > 0$. Zakładamy chwilowo, że aproksymujemy
funkcję na przedziale $[0, 2\pi]$, a liczba węzłów to $2L$.
\[
x_i = \frac{i\pi}{L}, \tab \text{dla}~~ i = 0, 1, \ldots, 2L - 1
\]
Aproksymacja funkcjami trygonometrycznymi polega na przybliżeniu
\[
f(x) \approx \frac{a_0}{2} + \sum_{i=1}^{K}\left(
a_i \cos(i \cdot x) + b_i \sin(i \cdot x)
\right)
\]
przy czym zakłada się, że $2K + 1 \leqslant 2L$.
\[
a_j = \frac{1}{L} \sum_{i=0}^{2L-1} y_i \cos(j \cdot x_i)
= \frac{1}{L} \sum_{i=0}^{2L-1} y_i \cos \frac{\pi i j}{L}
,\tab \text{dla} ~~j = 0, 1, \ldots, K
\]
\[
b_j = \frac{1}{L} \sum_{i=0}^{2L-1} y_i \sin(j \cdot x_i)
= \frac{1}{L} \sum_{i=0}^{2L-1} y_i \sin \frac{\pi i j}{L}
,\tab \text{dla} ~~j = 1, 2, \ldots, K
\]
\end{tcolorbox}

\pagebreak

\subsubsection{Przykład}
Wyznacz aproksymację trygonometryczną dla punktów
$
\left(  0, 3 \right)
$,
$
\left(  \frac{\pi}{3},  2 \right)
$,
$
\left(  \frac{2\pi}{3}, 3 \right)
$,
$
\left(  \pi, 1 \right)
$,
$
\left(  \frac{4\pi}{1}, 1 \right)
$,
$
\left(  \frac{5\pi}{3}, 2 \right)
$.
Mamy $2L = 6$ stąd $2K + 1 \leqslant 6$, czyli $L = 3$, $K = 2$. Zgodnie ze
wzorami wyznaczamy współczynniki.
\begin{align*}
a_0 &= \frac{1}{L} \sum_{i=0}^{2L-1} y_i \cos(0 \cdot x_i)
= \frac{1}{3} \sum_{i=0}^{5}y_i \cdot 1
= \frac{12}{3} = 4\\
a_1 &= \frac{1}{3} \left(
3\cos 0 + 2 \cos \frac{\pi}{3} + 3 \cos \frac{2\pi}{3} + \cos \pi
+ \cos \frac{4\pi}{3} + 2 \cos \frac{5\pi}{3}
\right) = \frac{2}{3}\\
a_2 &= \frac{1}{3} \left(
3\cos 0 + 2 \cos \frac{2\pi}{3} + 3 \cos \frac{4\pi}{3} + \cos 2\pi
+ \cos \frac{8\pi}{3} + 2 \cos \frac{10\pi}{3}
\right) = 0\\
b_1 &= \frac{1}{3} \left(
3\sin 0 + 2 \sin \frac{\pi}{3} + 3 \sin \frac{2\pi}{3} + \sin \pi
+ \sin \frac{4\pi}{3} + 2 \sin \frac{5\pi}{3}
\right) = \frac{\sqrt{3}}{3}\\
b_2 &= \frac{1}{3} \left(
3\sin 0 + 2 \sin \frac{2\pi}{3} + 3 \sin \frac{4\pi}{3} + \sin 2\pi
+ \sin \frac{8\pi}{3} + 2 \sin \frac{10\pi}{3}
\right) = -\frac{\sqrt{3}}{3}\\
\end{align*}
Ostatecznie otrzymujemy przybliżenie
\[
f(x) = \frac{4}{2} + \frac{2}{3} \cos x + \frac{\sqrt{3}}{3} \sin x
- \frac{\sqrt{3}}{3} \sin 2x
\]

\pagebreak
\end{document}
