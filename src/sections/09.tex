\documentclass[../mn-notatki.tex]{subfiles}

\begin{document}

\section{Funkcje sklejane}

\subsection{Naturalna funkcja sklejana}
\begin{tcolorbox}
Funkcję $s: \mathbb{R} \rightarrow \mathbb{R}$ nazywamy \textbf{naturalną
funkcją sklejaną} stopnia $2m + 1$ jeżeli
\begin{itemize}
    \item $s \in \Pi_{2m+1}$ w każdym z przedziałów $[x_i, x_{i+1}]$
    \item $s \in \Pi_{m}$ w $(\infty, x_{0}]$ oraz $[x_n, \infty)$
    \item $s \in \mathcal{C}^{2m}(\mathbb{R}$ czyli $s$ ma ciągłą $2m$-tą
    pochodną na całym $\mathbb{R}$.
\end{itemize}
\end{tcolorbox}

\begin{tcolorbox}
Naturalna funkcja sklejana stopnia $1$ jest funkcją ciągłą, kawałkami liniową,
która w przedziałach $(\infty, x_{0}]$ oraz $[x_n, \infty)$ jest funkcją stałą.\\

Naturalna funkcja sklejana stopnia $3$ w przedziałach
$(\infty, x_{0}]$ oraz $[x_n, \infty)$ jest funkcją liniową i dlatego druga
pochodna w punktach $x_0$ i $x_n$ zeruje się.
Otrzymujemy dwa dodatkowe warunki
\[
s''_0 (x_0) = 0 \tab s''_{n-1}(x_n) = 0
\]
Dlatego naturalna funkcja stopnia $3$ jest określona jednoznacznie.
\end{tcolorbox}

\begin{tcolorbox}
\[
\begin{cases}
f_i(x_{i+1}) = f_{i+1}(x_{i+1}) = y_{i+1} ~~~~~~ \forall i\\
f'_i(x_{i+1}) = f'_{i+1}(x_{i+1})  ~~~~~~ \forall i\\
f''_i(x_{i+1}) = f''_{i+1}(x_{i+1}) ~~~~~~ \forall i\\
f''_1(x_{1}) = f''_{n-1}(x_{n}) = 0 ~~~~~~ \Rightarrow \text{funkcja naturalna}\\
\end{cases}
\]

\end{tcolorbox}

\subsection{Naturalna funkcja sklejana stopnia trzeciego}
\begin{tcolorbox}
\[
h_i = x_{i+1} - x_i
\]
\[
u_i = 2(h_{i-1} + h_i)
\]
\[
b_i = \frac{6}{h_i}(y_{i+1}-y_i)
\]
\[
v_i = b_i - b_{i-1}
\]
\end{tcolorbox}

\begin{tcolorbox}
\[
\begin{bmatrix}
u_1 & h_1    &         &         &\\
h_1 & u_2    & h_2     &         &\\
    & \ddots & \ddots  & \ddots  &\\
    &        & h_{n-3} & u_{n-2} & h_{n-2} \\
    &        &         & h_{n-2} & u_{n-1}
\end{bmatrix}
\begin{bmatrix}
z_1 \\
z_2 \\
\vdots \\
z_{n-2}\\
z_{n-1}
\end{bmatrix}
=
\begin{bmatrix}
v_1 \\
v_2 \\
\vdots \\
v_{n-2}\\
v_{n-1}
\end{bmatrix}
\]
\end{tcolorbox}

Po obliczeniu wielkości $z_0, z_1, \ldots, z_n$ można znaleźć wartości funkcji
sklejanej $S$ w dowolnym punkcie $x$. Najpierw należy ustalić, do którego
z przedziałów
\begin{equation*}
(-\infty, x_1),
~~~~~~ [ x_1, x_2),
~~~~~~ \ldots,
~~~~~~ [x_{n-2}, x_{n-1}),
~~~~~~ [x_{n-1}, \infty)
\end{equation*}
$x$ należy. W tym celu badamy kolejno różnice
\[
x - x_{n-1}, \tab x - x_{n-2}, \tab \ldots, \tab x-x_1.
\]
Jeśli znajdziemy pierwszą z nich, np. $x - x_i$, która jest nieujemna, to
$x \in [x_i, x_{i+1})$. Jeśli wszystkie liczby są ujemne to $x \in (-\infty, x_1)$
i przyjmujemy $i = 0$. W ten sposób znajdujem wskaźnik właściwego wielomianu
$S_i$ i obliczamy jego wartość $S_i(x)$.

\begin{tcolorbox}
\[
S_i(x) = y_i + (x-x_i)\left(  C_i + (x-x_i) (B_i + (x-x_i) A_i ) \right)
\]
\end{tcolorbox}
\begin{tcolorbox}
\[
A_i := \frac{1}{6h_i}(z_{i+1}-z_i)
\]
\[
B_i := \frac{z_i}{2}
\]
\[
C_i := -\frac{h_i}{6}(z_{i+1} + 2z_i) + \frac{1}{h_i} (y_{i+1} - y_i)
\]
\end{tcolorbox}


\pagebreak
\end{document}
