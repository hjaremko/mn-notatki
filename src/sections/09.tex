\documentclass[../mn-notatki.tex]{subfiles}

\begin{document}

\section{Funkcje sklejane}

\subsection{Naturalna funkcja sklejana stopnia trzeciego}
\begin{tcolorbox}
\[
h_i = x_{i+1} - x_i
\]
\[
u_i = 2(h_{i-1} + h_i)
\]
\[
b_i = \frac{6}{h_i}(y_{i+1}-y_i)
\]
\[
v_i = b_i - b_{i-1}
\]
\end{tcolorbox}

\begin{tcolorbox}
\[
\begin{bmatrix}
u_1 & h_1    &         &         &\\
h_1 & u_2    & h_2     &         &\\
    & \ddots & \ddots  & \ddots  &\\
    &        & h_{n-3} & u_{n-2} & h_{n-2} \\
    &        &         & h_{n-2} & u_{n-1}
\end{bmatrix}
\begin{bmatrix}
z_1 \\
z_2 \\
\vdots \\
z_{n-2}\\
z_{n-1}
\end{bmatrix}
=
\begin{bmatrix}
v_1 \\
v_2 \\
\vdots \\
v_{n-2}\\
v_{n-1}
\end{bmatrix}
\]
\end{tcolorbox}

Po obliczeniu wielkości $z_0, z_1, \ldots, z_n$ można znaleźć wartości funkcji
sklejanej $S$ w dowolnym punkcie $x$. Najpierw należy ustalić, do którego
z przedziałów
\begin{equation*}
(-\infty, x_1),
~~~~~~ [ x_1, x_2),
~~~~~~ \ldots,
~~~~~~ [x_{n-2}, x_{n-1}),
~~~~~~ [x_{n-1}, \infty)
\end{equation*}
$x$ należy. W tym celu badamy kolejno różnice
\[
x - x_{n-1}, \tab x - x_{n-2}, \tab \ldots, \tab x-x_1.
\]
Jeśli znajdziemy pierwszą z nich, np. $x - x_i$, która jest nieujemna, to
$x \in [x_i, x_{i+1})$. Jeśli wszystkie liczby są ujemne to $x \in (-\infty, x_1)$
i przyjmujemy $i = 0$. W ten sposób znajdujem wskaźnik właściwego wielomianu
$S_i$ i obliczamy jego wartość $S_i(x)$.

\begin{tcolorbox}
\[
S_i(x) = y_i + (x-x_i)\left(  C_i + (x-x_i) (B_i + (x-x_i) A_i ) \right)
\]
\end{tcolorbox}
\begin{tcolorbox}
\[
A_i := \frac{1}{6h_i}(z_{i+1}-z_i)
\]
\[
B_i := \frac{z_i}{2}
\]
\[
C_i := -\frac{h_i}{6}(z_{i+1} + 2z_i) + \frac{1}{h_i} (y_{i+1} - y_i)
\]
\end{tcolorbox}


\pagebreak
\end{document}
